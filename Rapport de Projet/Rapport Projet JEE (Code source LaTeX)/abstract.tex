\renewcommand{\abstractnamefont}{\normalfont\Large\bfseries}
%\renewcommand{\abstracttextfont}{\normalfont\Huge}

\begin{abstract}
\hskip7mm

\begin{spacing}{1.3}
Des ventes de main en main, vers des ventes virtuelles, passent les priorités des opérations de ventes des biens et des services, ce qui nous rend obligés à donner plus d’importance à la vente électronique.
Les boutiques en ligne sont depuis des années, largement conseillées pour les entreprises qui se basent sur la vente des produits, et même des services. Ces types de sites web représentent un dispositif global fournissant aux clients un pont de passage à l’ensemble des informations, des produits, et des services à partir d’un portail unique en rapport avec son activité.
Les sites de vente en ligne permettant aux clients de profiter d’une foire virtuelle disponible sont quotidiennement mises à jour sans la moindre contrainte, ce qui leur permettrait de ne jamais rater les coups de cœur, ainsi une foire sans problème de distance géographique, ni d’horaire de travail ni de disponibilité de transport. D’une autre par ces sites offre à l'entreprise de profiter de cet espace pour exposer ses produits à une plus large base de clientèle.
Notre projet est de réaliser dans le cadre du projet java EE de deuxième année GL ayant comme objectif principal : la conception et la réalisation d'une E-boutique de livre. Ce rapport est composé de trois volets, le premier contient le cahier de charges. Dans le deuxième, nous décrivons l’analyse et la conception de notre application, et dans le troisième nous présentons les outils utilisés pour la réalisation ainsi que des captures d’écran de l’application avec description.
\end{spacing}
\end{abstract}
