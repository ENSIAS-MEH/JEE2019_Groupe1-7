
\chapter{Conclusion }
\addcontentsline{toc}{section}{Conclusion }


En réalisant un projet de vente à distance, on a permis au consommateur, en dehors des lieux habituels de réception de la clientèle, d'effectuer sa commande. Cela était possible grâce aux connaissances acquises durant notre formation a l'ENSIAS.
\newline
L’utilisation de l'UML nous a permis d’effectuer une analyse détaillée de l'e-boutique de livre. Ainsi, nous avons pu parfaire nos compétences en termes d’analyse et de conception.
L'utilisation des Javabeans et des DAO s'avérait indispensable pendant la rédaction du code.
\newline
Au niveau de la gestion du projet en binôme, nous avons réussi à bien nous répartir les tâches afin de réaliser nos objectifs dans le temps qu'on avait, Github était incontournable en ce qui concerne ce volet.
En guise de perspectives, il serait intéressant d’ajouter des modes de paiement à notre application web e-boutique de livre comme Paypal et cartes bancaires.